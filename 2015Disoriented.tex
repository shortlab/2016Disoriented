%% LyX 2.0.6 created this file.  For more info, see http://www.lyx.org/.
%% Do not edit unless you really know what you are doing.
\documentclass[english]{article}
\usepackage[OT1]{fontenc}
\usepackage[latin9]{inputenc}
\usepackage{geometry}
\geometry{verbose,tmargin=1in,bmargin=1in,lmargin=1in,rmargin=1in}
\usepackage{units}
\usepackage{babel}
\begin{document}

\title{Building the Bridge to Predictive, Mesoscale Radiation Materials
Science: An Example from Transient Grating Spectroscopy}


\author{C. Dennett, P. Cao, A. Vega-Flick, S. E. Ferry, A. A. Maznev, K.
A. Nelson, M. P. Short}
\maketitle
\begin{abstract}
Understanding and predicting mesoscale material properties requires
a synchronzied effort among theory, simulations, and experiments.
Predictive simulations represent the way to guiding and understanding
mesoscale experiments, as theories of mesoscale material response
have yet to mature. We show the application of transient grating (TG)
spectroscopy as an illustrative method to bridge the gap towards understanding
mesoscale radiation-induced material property changes. Classical,
non-equilibrium molecular dynamics (MD) simulations are shown to accurately
predict theoretically-predicted and experimentally-measured changes
in Young's modulus due to small variations in anisotropic single crystal
orientation, at levels similar to those induced by irradiation. With
the ability of predictive MD to simulate the TG signal validated by
both theory and experiment, it can be used as a tool to both guide
and understand TG experiments of materials with defect populations,
which current theories cannot yet describe. In addition, the results
illustrate the usage of TG spectroscopy to rapidly determine unknown
single crystal orientations.
\end{abstract}

\section{Introduction}

The mesoscale frontier of science is ripe with opportunities for discovery
{[}REF - Crabtree and Sarrao{]}, with the ultimate goal of linking
atomistic, unit processes with component-level performance. Since
its inception {[}REF - Middle Way{]}, those in the growing field of
mesoscale science (MSS) seek to understand the mechanistic reasons
for the evolution of material properties not just from defects, but
also from defect populations, superstructures, and spatiotemporal
variations. However, significant difficulty arises in the direct theoretical
prediction and the experimental measurement of mesoscale properties
of materials. Theory often treats response functions, such as that
of elasticity or thermal conductivity, as a superposition of a homogeneous
medium with a mathematically convenient distribution of single defects.
Analytical treatment of a complex material, containing a wide variety
of defect types, sizes, and spatial distributions, is beyond mesoscale
theory at the moment. In these cases experimental measurements can
yield insights to elucidate underlying theories, though direct mesoscale
measurements continue to prove difficult to perform, or even conceive.

Examples of emerging mesoscale fields exist throughout the literature,
ranging from strain rate-dependent flow in amorphous materials {[}REF{]},
to stress corrosion cracking {[}REF{]}, to {[}MORE{]}. Radiation damage
provides an excellent example of the need for such an approach, as
no unified theory can yet predict the response of an arbitrary material
to radiation damage. In fact, the standardized unit of measure, the
displacements per atom (DPA), is not a measurement, but a calculated
exposure parameter. It does not account for the differing defect populations
(point defects, dislocations, voids, precipitates, stacking faults...)
responsible for the determination of material properties of interest,
reflected in multiple datasets of differing radiation damage response
at identical DPA levels, with different environmental parameters {[}REFs{]}.
In this case, directly coupled mesoscale predictions and measurements
are needed to understand and predict radiation's effects on material
property changes.

Ideally, one would develop experimental ideas based on a basis of
theory, and then use simulations to understand the mechanism of the
response. Another potential path for success is to use simulations
as a predictive tool, validated by theory, to simultaneously guide
and understand experimental measurements. For a field without a fully
described theoretical framework, such as mesoscale radiation materials
science, this second path is the only path. However, it cannot proceed
without first rigorously validating the simulation framework using
theoretical predictions and well-understood experimental measurements.

One promising tool for direct, mesoscale measurements is transient
grating (TG) spectroscopy \cite{Rogers2000}. This technique projects
a sinusoidal pattern of laser light onto a material on the order of
100 microns in width, allowing for direct mesoscale measurement of
elastic constants, thermal conductivity, and acoustic damping of surface
acoustic waves (SAWs). Its ability to discern changes in material
properties due to underlying defect populations depends strongly on
the repeatable sensitivity of the measurement itself. Because these
material properties can be anisotropic in solid materials, and this
anisotropy can grow tremendously with irradiation {[}REF{]}, knowing
the precise surface-normal orientation \emph{and} relative rotation
of the crystal face are critical to the sensitivity of the measurement.
The nature of most of the measurements made using TG spectroscopy
in the past has precluded the need for such detailed, careful studies.
Most studies have focused on determining thermal transport in transparent
liquids {[}REF{]}, or SAW propagation on elastically isotropic solids
\cite{Hofmann2015,Hofmann2015a}. Most studies simply do not account
for surface rotation in the measurements, instead relegating resultant
measurement variations to experimental error {[}REFs{]}. Even the
quality of the surface makes an enormous difference, with 8-27\% differences
in SAW speed reported on common steels as a function of the method
of surface preparation \cite{Kitazawa2015}.

These approaches may have worked for the quantities measured in previous
studies, but in radiation materials science, quantities like the Young's
modulus must be reliably determined to 100's of parts per million
(ppm). This is due to the fact that radiation-induced defects change
elastic constants in well-defined, though small, amounts. Vacancies
have been analytically \cite{Dienes1952,Nabarro1952,Dienes1952a}
and computationally \cite{Chiesa2009} shown to change elastic constants
on the order one percent per percent concentration. Interstitial atoms
are predicted to have an effect between two \cite{Chiesa2009} and
ten \cite{Dienes1952,Nabarro1952,Dienes1952a} times as strong in
magnitude. These defects are found in levels from $\mathrm{10^{-10}}$
to $\mathrm{10^{-4}}$ atom fraction for vacancies from room temperature
to the melting point, respectively, and at levels thousands of times
lower for interstitials. Radiation damage can increase vacancy concentrations
to a theoretical maximum of one percent during intense irradiation
\cite{Averback1977}. Dislocations change the Young's modulus via
Equation \ref{eq:Dislocation-Youngs-Modulus-Change}:
\begin{equation}
\frac{\Delta E}{E}=\frac{\pi k\rho_{\bot}L^{2}}{3\, ln\left(\frac{L}{\left|\overrightarrow{b}\right|}\right)\,}\label{eq:Dislocation-Youngs-Modulus-Change}
\end{equation}
where \emph{E} is the Young's modulus, $\Delta E$ is its change,
$\rho_{\bot}$ is the dislocation density, L is the average dislocation
pinned segment length, and k is unity for a screw dislocation, and
$\left(1-\nu\right)$ for an edge dislocation, where $\nu$ is the
Poisson's ratio \cite{Friedel1953}. Voids are expected to reduce
the Young's modulus by a magnitude on the order of their volume fraction.
In all cases, resolving changes in the Young's modulus well below
1\% is critical to the usage of TG spectroscopy as a mesoscale measurement
technique for radiation damage.

Measurement of the anisotropy of the Young's modulus provides a convenient
method of validating both the experimental technique and the simulation
approach at reliably discerning such small variations in Young's modulus.
Here anisotropy is defined by the Zener anisotropy ratio $\left(A\right)$
in Equation \ref{eq:Anisotropy ratio}:
\begin{equation}
A=\frac{2C_{44}}{\left(C_{11}-C_{12}\right)}\label{eq:Anisotropy ratio}
\end{equation}
where $\mathrm{C_{ij}}$ is the elastic constant in the ith row and
the jth column of the elasticity tensor \cite{Zener1947}. Therefore,
by simply rotating a single crystal of a material with a known elastic
anisotropy, one can tune the expected change in Young's modulus from
theory, while simultaneously measuring and predicting it using TG
and MD, respectively. The ultimate test is to select materials with
a very low degree of anisotropy, like aluminum, which exhibits the
second lowest anisotropy ratio $\left(A=1.22\right)$ only to tungsten
$\left(A=1.01\right)$ \cite{Simmons1971}, resulting in expected
changes in SAW speed of fractions of a percent per degree of rotation.
Should the TG measurements made on materials like aluminum be discernible,
they will simultaneously qualify the technique for the study of fine
changes in Young's modulus, and set a lower bound on the noise-based
sensitivity of the TG spectroscopy technique. It should also be noted
that irradiation has been shown to increase the anisotropy of \_\_\_
{[}REF{]}, helping to more confidently transition the application
of TG spectroscopy from pure materials to irradiated materials.

In this manuscript, we show the application of transient grating (TG)
spectroscopy to measuring one quantity of interest in radiation materials
science, the Young's modulus, within the limits of sensitivity for
expected changes due to irradiation. We then hypothesize that classical
molecular dynamics (MD) is an appropriate tool for simulating the
linear elastic response of materials to the type of excitation used
in transient grating spectroscopy for single crystal metals, therefore
making it an appropriate predictive tool. Two model materials, pure
Al and Cu single crystals, are chosen to elucidate the sensitivity
limits of such a measurement by rotating the crystal, and measuring
orientation-dependent Rayleigh wave speeds from projected TGs. Then,
classical MD simulations are used to simulate the TG spectroscopy
signal on identical materials at different orientations. These combined
experimental and simulated results are plotted on theoretically predicted
slowness surfaces using the elastodynamic Green's function {[}REF{]},
showing excellent agreement. Scaled-down MD simulations are shown
to predict larger-scale TG response, setting a lower limit of simulation
scale applicable to predicting TG measurements. With theory, simulation,
and experiment in agreement, work can now proceed to predict and measure
materials with defect populations, forging ahead into an area without
a full theoretical underpinning, but with confidence in mesoscale
simulations as a quantitative, predictive tool.

In addition, the utility of TG spectroscopy to determine unknown crystal
orientations by similar rotational measurements is demonstrated, as
a potentially more rapid and accessible alternative to Laue backscatter
diffraction or electron backscatter diffraction (EBSD).


\section{Methods}

In the following sections, examples of the measurement, simulation,
and analysis methods are given for expanation. All raw data files,
processing scripts, output datasets, simulation input files, intermediate
relaxed atom files, final atomic configurations, compiled/processed
datafiles, and the code used to generate the slowness surfaces with
input parameters are hosted permanently on our GitHub repository {[}REF{]}.


\subsection{Experimental Procedure (Figure 1a)}
\begin{itemize}
\item Procurement \& specs for W, Al, and Cu single crystals, confirmation
method(s) for orientation of faces, methods of surface preparation
\item Minimal, yet functional, description of the TG facility, offload what
you can to the JOM paper's description

\begin{itemize}
\item Laser power, rep rate, spot size, grating spacing, and sample management
(rotation stage) parameters detailed
\item Include heterodyne discussion, offload what you can to the Maznev
Opt. Lett. paper
\end{itemize}
\item Calibration procedure with tungsten - be quite specific here, wasn't
mentioned as much in the JOM paper
\item Rotation \& measurement procedure for each material, mention repetition,
limits of measurement uncertainty from signal noise floor, knowledge
of specimen rotation on the stage (half the smallest gradation)
\item Detailed explanation with equations of the data decomposition process,
as well as analyzing the frequency power spectrum for peak frequency.
Mention FWHM added to the uncertainty of the measurement.

\begin{itemize}
\item Heterodyne phase correcting, non-exponential thermal decay subtraction,
derivative taking, periodogram taking, peak finding, and velocity
calculations from measured calibration sample from each run
\end{itemize}
\end{itemize}

\subsection{Simulation Procedure (Figure 1b)}
\begin{itemize}
\item First mention that we used LAMMPS, software versions, potentials used
with parameters and references (and citations for validation of these
potentials)
\item Quickly mention high-level process:

\begin{itemize}
\item Automatically build (001) and (111) surfaces of each material in LAMMPS,
NPT-relax to \_\_\_ timesteps (or \_\_\_ time)
\item Feed relaxed data into laser simulation, and track the motion of surface
atoms during an NVE relaxation
\end{itemize}
\item Next, show a schematic and explanation of the TG projection simulation
process

\begin{itemize}
\item Mention breaking periodicity in z-direction to study wave propagation,
with quantitative numbers
\item Mention x/y periodic boundary conditions
\item Show schematic and temperature function describing the heating pattern,
emphasize that it's time-dependent too
\begin{equation}
T\left(x,z,t\right)=T_{0}\left(1+\left(\left|x\right|<x_{0}\right)\frac{t}{t_{0}}e^{\nicefrac{-z}{z_{0}}}\right)
\end{equation}


\begin{itemize}
\item Talk about how/why we tuned values of $\mathit{t_{0},\, x_{0},\, z_{0}}$
to get the best data - we will be asked!
\item Mention that adding an x-Gaussian was also tried, but made no difference
\end{itemize}
\item Show how simulation data (COMALL{[}3{]} - COM1x{[}3{]}) can be analyzed
in precisely the same way, using the same scripts, as the experimental
data (include Sara's/Mike's script from NiAl)
\end{itemize}
\item Finally, show the test matrix (sizes tested, number of atoms, timesteps
used, etc. since some didn't completely finish)
\end{itemize}

\subsection{Theoretical Slowness Surface Prediction}
\begin{itemize}
\item Need to find a source where Every puts forth all the theory

\begin{itemize}
\item Based on the methodology presented by Farnell in Physical Acoustics
vol. VI
\item Hopefully also a source where he explains/demos the code
\end{itemize}
\item Mention how it works, and what the various lines mean in terms of
valid/likely modes
\end{itemize}

\section{Results}


\subsection{Experimental Results}
\begin{itemize}
\item Mention calibration data in the next, no need for a figure.
\item Figure 2: Raw data showing SAW speeds with rotation for two different
surfaces, on two different materials
\end{itemize}

\subsection{Simulation Results}
\begin{itemize}
\item Table 1: Tabulated simulation data, since there isn't nearly as much
as the experimental stuff
\end{itemize}

\subsection{Combined Results}
\begin{itemize}
\item Figure 3 (the big one): Quarter slowness surfaces for each material
and surface orientation, with theory/experiment/simulation all together
\item Figure 4 (the other big one): Combined experiment/theory/simulation
graphs, showing correlation bewteen expt. \& simulation, graph of
\% error below, explanation of why no error bars (too small to see),
and a cutoff showing where we believe the ``range of validity''
to be with MD
\end{itemize}

\section{Discussion}


\section{Conclusion}

\bibliographystyle{unsrt}
\bibliography{2015Disoriented}

\end{document}
