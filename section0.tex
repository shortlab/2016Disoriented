%Abstract

The outline content contained in section1-6 is based on our first discussion of the topic of the TG aluminum orientation paper. Since then, the idea has developed somewhat and the motivation for the study that we are now conducting has been updated to fit in with the overall research scheme that we outlined on the 6th of August, 2015. The following are notes that Cody tool directly after that meeting and which could serve as a basis for the continued development of this work:

Here's the scoop after meeting with Mike. This is all still tied back to the radiation damage problem. The physics behind anisotropic elastic repsonse theory has been known for some time. Similar measurments to the ones that we're working on showing that this is in fact a real effect in single crystal metals has been done for pretty anisotropic materials (see the Cu study from '68). What we're doing that's new is that we can make these measurements with a relatively high degree of precision on materials that are not very anisotropic, Al is one of the closest to tungsten, and we can also get the same measurable changes using molecular dynamics simulations. MD is the key because when looking towards mesoscale problems, you wont be able to plug into analytic formulae to make predictions, you'll need some heterogeneity in the things that you're using to make predictions on measurements that you can't do with the analytical tools at present. If, and since we're seeing this is the case, you can get these same very small scale changes using MD simulations, then they may in fact be the MOST appropriate tool to use in extending this technique for many applications.