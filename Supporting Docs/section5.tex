%Discussion

\begin{itemize}
\item Comparison of experiment, theory, and simulation
	\begin{itemize}
	\item This can be best accomplished in a table which lists the relative change in SAW speed with orientation, the biggest percent variation, from theory, experiment and simulation. Make sure to note the lengthscales used for experiment and simulation.
	\item Note that to reasonable agreement everything works out, all of the variations are on the same percent scale, and maybe even work in a plot once we can normalize for actual speeds plotting all these against each other. $\leftarrow$ That would be really interesting to see for one case.
	\end{itemize}
\item The rest of the discussion points have already sort of been mentioned under ``Use'' that is in the intro now.
\item Unknown crystal orientation
	\begin{itemize}
	\item We again show the use of this technique to determine unknown single crystal orientations coupled with theoretical predictions of the SAW velocity. However, what we get now is the ability to make the same sorts of predictions using simulations and not theoretical predictions. This is the key, because theoretical predictions rely on elastic parameters as inputs and if you have a material with some manner of defect population that will change the effective elastic parameters, then you can no longer rely on theoretical predictions to allow you to accurately predict orientation. The kicker is that using MD simulations, you can in fact account for the possibility of defect populations and allow yourself to being to identify classes of non-ideal materials.
	\end{itemize}
\item MD as an appropriate tool
	\begin{itemize}
	\item Similarly to the point above, we see that MD simulations used in this way are sensitive enough to show these single percent changes in acoustic properties and effectively return the same information as experiment. This indicates that MD may in fact be THE most appropriate tool to explore the effects of mesoscale defects on the SAW responses of materials.
	\item If one is interested, like we are in the field of radiation damage, in studying the effects of single and multiple classes of defects on the elastic properties of materials, then MD seems to be a complimentary tool for the cases in which theory with \emph{a priori} knowledge of elastic constants fails. Especially in the case of the reverse problem, when you are seeking to explore the effects of mesoscale defects on the elastic constants.
	\item This maybe should go before ``Unknown crystal orientations''.
	\end{itemize}
\item Bounds on experimental uncertainty
	\begin{itemize}
	\item Although important, I'm not sure if a detailed discussion of this point fits in with the scope of this work, besides a mention of how we're handling uncertainly and its propagation in the experiment section. 
	\item It may be best to use this paper as a reference in the future for a beginning on the bounds of what we can detect and since we're not talking specifically about radiation damage effects on the sub-percent scale yet, we may not need to mention anything explicitly.
	\end{itemize}
\end{itemize}

% \begin{itemize}
% \item Comparing the MD and experimental data
% \begin{itemize}
% \item Implications for ease of determining the orientation of unknown crystal substrates
% \end{itemize}
% \item More anisotropic materials should be easier
% \begin{itemize}
% \item Is there theory anywhere that would let us make predictions about differing speeds on not only crystal faces but also directions on those faces? I want to know this anyway
% \end{itemize}
% \item Uncertainty
% \begin{itemize}
% \item Can talk about here or just find a convincing way to handle experimental uncertainty and include that in the results section, leaning towards the latter
% \end{itemize}
% \item New method for qualification of MD potentials?
% \end{itemize}