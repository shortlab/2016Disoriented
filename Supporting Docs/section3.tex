%Methods

\begin{itemize}
\item Material Choice
	\begin{itemize}
	\item Aluminum: Focused efforts of this study on Al since it is a relatively isotropic material. If we are sensitive to the small changes in SAW velocity expected with relative orientation change on the surface in both experiment and simulation then that is a strong indicator of simulation performance.
	\item Tungsten: Chosen as an experimental calibration standard, at \{100\}, since it is the most isotropic simple metal. Expect and see little variation experimentally at different orientations along the surface. 
	\item Be sure to mention surface preparation for both of the experimental samples. 
	\item Copper (?): Chosen as additional simulation case since it is a more anisotropic material and shows that potentials for a variety of materials are available and appropriate to make the kinds of predictions we're interested in.
	\end{itemize}
\item Experimental Methodology
	\begin{itemize}
	\item Experiment description including discussion of heterodyne detection for signal amplification. This include a \textbf{diagram of the experimental setup}, which may need to be modified after I finish re-aligning the setup. 
	\item Power, rep rate, spot size, grating spacing, and sample management (rotation stage) parameters detailed.
	\item Data processing: Heterodyne phase correcting, non-exponential thermal decay subtraction, derivative taking, periodogram taking, peak finding, and velocity calculations from measured calibration sample from each run.
	\end{itemize}
\item Molecular Dynamics Methodology
	\begin{itemize}
	\item Simulation volume construction including various sizes under investigation.
	\item Discussion of potential that is used for the simulations.
	\item Excitation method and similarities to experimental excitation.
	\item Relative orientation changes and how they're implemented within the framework of corresponding to experiment. \textbf{This is where a figure of both the simulation volume and how its oriented would be helpful.}
	\item Data recovered (z-tracking) and processing procedure, sensitivity to motion orthogonal to the surface only. 
	\end{itemize}
\item Theoretical Predictions
	\begin{itemize}
	\item Based on the methodology presented by Farnell in Physical Acoustics vol. VI. 
	\item Need to hear back from A. G. Every before I can more accurately write to this point.
	\end{itemize}
\end{itemize}

% \begin{itemize}
% \item Chose Al to study because it is relatively isotropic, if you can detect changes with it then more anisotropic materials should be easier
% \item Description of the TG methodology
% \item Calibration with tungsten (as we see it doesn't change with rotation)
% \item Three figures
% \begin{itemize}
% \item Photo of TG facility
% \item More advanced diagram (solidworks from Mike?)
% \item TG analysis image like quals
% \end{itemize}
% \item Description of analysis methodology
% \item MD Methodology
% \begin{itemize}
% \item Whatever Penghui thinks is necessary in here
% \end{itemize}
% \end{itemize}